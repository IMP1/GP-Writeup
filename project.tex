\documentclass{UoYCSproject}
\author{Huw Taylor}
\title{Tracing and Debugging GP2}
\date{\today}
\supervisor{Dr. Detlef Plump}
\MEng
\wordcount
\pagecount
\abstract{
...
}
\acknowledgements{

}

\begin{document}

\maketitle
\tableofcontents
\listoffigures

\chapter{Introduction}
\section{Motivation}
The University of York has produced a graphical editor for creating host graphs and graph programmes. The editor depends on a compiler to produce the resultant graph. Both these tools are planned to be used by students in the coming academic year (2016/2017) for the GRAT module.
The aim of this project is to show the intermediate steps of the programme, so that correctness of the compiler and the graphical editor can be ensured, and also so that students using these tools to learn how graph programmes work can see the process with a finer granularity.
\section{Ethics}
The project discussed has very few ethical considerations. It is not related to defence aand there are no safety or security concerns.

\chapter{Literature Review}


 * To show that you know what is happening in your field 
 * To justify why your work is interesting or important 
 * To establish the theoretical framework/context for your work 
 * To defend your choice of methodology 
 * To avoid repeating previous researchers’ mistakes
\section{Graph Programming}
A graph is a visual way of representing data and relationships. The formal definition is a set of vertices (nodes) \emph{V}, a set of edges \emph{E}, and a set of labels \emph{L}. Additionally \emph{source} and \emph{target} functions associate edges with nodes, and a \emph{label} function, which maps labels to edges and nodes.

\subsection{Graph Transformations}
Mathematical theory of graphs allows us to transform them.
\subsection{GP}
Plump \& Steinart
\subsection{GP2}
Elliot 5.1 (p29-30)
Bak 5.2 (p67)
\section{Graphical GP Editors}

\subsection{GP Editor}
Zhang 2.3.2 (p31-32)
Onder 3.2 (p15)
\subsection{GP2 Editor}
Previous masters students stuff
Ivaylo
\section{Tracing}

\chapter{Approach}
\section{Requirement Identification}
\begin{enumerate}
	\item Rules must be able to be stepped through one at a time.
 	\item The current rule must be made apparent graphically somehow.
	\item \ldots
\end{enumerate}
\section{Design}
 * Lifecycle
 * Engineering approach
 * Time management and/or general plan
 
\subsection{Tools}
Not a lot really.
\begin{itemize}
  \item Git
  \item Some C compiler/IDE/something or another
  \item Dependencies for GP-Editor
  \item \ldots
\end{itemize}
\section{Implementation}
\section{Evaluation}

\chapter{Evaluation}
\chapter{Conclusions}
\section{Future Work}
\chapter{Glossary}
\begin{description}
  \item[LHS] \hfill \\
  An initialism for Left Hand Side, \ldots
  \item[RHS] \hfill \\
  An initialism for Right Hand Side, \ldots
  \item[\ldots] \hfill \\
  \ldots
\end{description}

% \bibliography{references}

\chapter{Appendices}
\end{document}