\documentclass{UoYCSproject}
\author{Huw Taylor}
\title{Tracing and Debugging GP2}
\date{\today}
\supervisor{Dr. Detlef Plump}
\MEng
\wordcount
\pagecount
\abstract{
...
}
\acknowledgements{
Detlef Plump
Christopher Bak
Ivaylo Hristakiev
...?
}

\begin{document}

\maketitle
\tableofcontents
\listoffigures

\chapter{Introduction}
\section{Motivation}
\section{Ethics}

\chapter{Literature Review}
 * To show that you know what is happening in your field 
 * To justify why your work is interesting or important 
 * To establish the theoretical framework/context for your work 
 * To defend your choice of methodology 
 * To avoid repeating previous researchers’ mistakes
\section{Graph Programming}
\subsection{Graph Transformations}
A graph is a very visual way of representing data and relationships. 
Mathematical theory of graphs allows us to transform them.
\subsection{GP}
Plump & Steinart
\subsection{GP2}
Elliot 5.1 (p29-30)
Bak 5.2 (p67)
\section{Graphical GP Editors}

\subsection{GP Editor}
Zhang 2.3.2 (p31-32)
Onder 3.2 (p15)
\subsection{GP2 Editor}
Previous masters students stuff
Ivaylo
\section{Tracing}

\chapter{Approach}
\section{Requirement Identification}
\begin{enumerate}
  \item Rules must be able to be stepped through one at a time.
  \item The current rule must be made apparent graphically somehow.
  \item \ldots
\end{enumerate}
\section{Design}
 * Lifecycle
 * Engineering approach
 * Time management and/or general plan
 
\subsection{Tools}
Not a lot really.
\begin{itemize}
  \item Git
  \item Some C compiler/IDE/something or another
  \item Dependencies for GP-Editor
  \item \ldots
\end{itemize}
\section{Implementation}
\section{Evaluation}

\chapter{Evaluation}
\chapter{Conclusions}
\section{Future Work}
\chapter{Glossary}
\begin{description}
  \item[LHS] \hfill \\
  An initialism for Left Hand Side, \ldots
  \item[RHS] \hfill \\
  An initialism for Right Hand Side, \ldots
  \item[\ldots] \hfill \\
  \ldots
\end{description}

% \bibliography{references}

\chapter{Appendices}
\end{document}