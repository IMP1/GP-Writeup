\begin{figure}
\label{fig:simple_program}
\centering
\begin{tikzpicture}[scale=2]
  \node (code) at (0, 3.5) {$main = InitTags!; Reduce$};
  \node (rule_1) at (0, 2.8) {$InitTags(a,b,c,n : int) = $};

  \node (n1) [vertex, label=below:{1}] at (0, 2.1) {$a_n$};
  \node (n2) [vertex, label=below:{2}] at (1, 2.1) {c};
  \path [edge] (n1) edge node[label=above:{b}] {} (n2);
  \node (r1_l) [rule, fit={(n1) (n2)}] {};
  
  \node (n3) [vertex, label=below:{1}] at (2, 2.1) {$a_n$};
  \node (n4) [vertex, label=below:{2}] at (3, 2.1) {$c_n+b$};
  \path [edge] (n3) edge node[label=above:{b}] {} (n4);
  \node (r1_r) [rule, fit={(n3) (n4)}] {};
  
  \draw[derivation] (r1_r) -> (r1_l); % dafuq is this backwards bullshit?
  
  \node (rule_2) at (0, 1.5) {$Reduce(a,b,c,n,i : int) = $};  
  
  \node (n5) [vertex, label=below:{1}] at (0, 0.8) {$a_n$};
  \node (n6) [vertex, label=below:{2}] at (1, 0.8) {$c_i$};
  \path [edge] (n5) edge node[label=above:{b}] {} (n6);
  \node (r2_l) [rule, fit={(n5) (n6)}] {};
  
  \node (n7) [vertex, label=below:{1}] at (2, 0.8) {$a_n$};
  \node (n8) [vertex, label=below:{2}] at (3, 0.8) {$c_n+b$};
  \path [edge] (n7) edge node[label=above:{b}] {} (n8);
  \node (r2_r) [rule, fit={(n7) (n8)}] {};
  
  \draw[derivation] (r2_r) -> (r2_l); % ~EDIT: Maybe use the deriviation tikz thing defined at the top.
  \node (rule_2_condition) at (0, 0) {$where n+b < i$};  

\end{tikzpicture}
\caption{Simple Dijkstra Program Example}
\end{figure}